\chapter{Summary \& Future Work}\label{chap:summary_and_future_work}


%SUMMARY%
%1.st paragraph: briefly describe the motivation and approach									DONE
%2.nd paragraph: briefly describe proposed methods												DONE
%3.rd paragraph: briefly describe the results													DONE

%FUTURE WORK%
%4.th ff paragraphs: describe big pricture														DONE
	% Proposed solution is embedded into bigger context (e.g. ontology learning solution)		DONE
	% Provides only a tiny poprtion																DONE

%paragraph: add transition text to open questions --- future tasks								DONE
%paragraph: for each open task add a new paragraph here											DONE

In this thesis we investigated whether contextual information in Crowdsourcing tasks helped to achieve better results for performing ontology validation using Crowdsourcing techniques. Crowdsourcing is a technique of distributing small tasks to a typically large group of human workers. It offers a cost effective method of solving tasks which are traditionally hard for machines but easily solvable by humans. Contextual information is any kind of of additional information that is supplied with a crowdsourcing task that improves the understanding of the task goals.

We presented three novel methods that enrich Crowdsourcing tasks with contextual information to validate the relevance of concepts for a particular domain of interest. Whereas the Ontology~based~Approach processes hierarchical relations, the Metadata~based~Approach generates descriptions based on annotations that were encoded within the ontology, the idea of the Dictionary~based~Approach is to form the explanations from example sentences by conducting the online dictionary \hyperref[sec:wordnik]{WordNik}.

All three approaches that were evaluated on three ontologies covering the domains of climate change, tennis and finance worked better than providing no Context at all. The Metadata~based~Approach outperformed all other methods in terms of precision and recall, leaving little room for future improvements. The other two approaches had some difficulties in certain situations, for example the Dictionary~based~Approach sometimes added inappropriate explanations, especially for concepts with multiple meanings associated. Likewise, the Ontology~based~Approach is limited to highly connected ontologies containing many subsumption relations. 

Before discussing open questions that needs investigation in future work, we take a look at the beginnings of this thesis. We explain, how our contributions relate to the overarching vision of providing an ontology learning solution which beats current implementations. 

It started by the publication~\cite{liu2005semi} which described the initial ontology learning framework. It generates the target ontology by conducting various sources to continuously extend a seed ontology. The framework underwent multiple revisions~\cite{weichselbraun2010_a, weichselbraun2010_b} until validation capabilities using Crowdsourcing techniques were integrated into the framework~\cite{wohlgenannt2012}.  
At that time human evaluation of learned ontologies was just at the beginning, leaving much room for improvements. 

Then the question arises, how the results of this thesis can contribute to this vision. This unexplored topic needs to be addressed by future research, possibly only after the improvement suggestions that are related to the approach taken by this thesis are in place. 

While this work tried to advance research, covering the intersection between Semantic Web technologies and Crowdsourcing, there are still several open questions that could not be addressed, because these new questions that are discussed in the remainder of this chapter arose during the course of this work.

Unfortunately, the Dictionary~based~Approach had problems in situations when multiple meanings were associated with the same concept or names contained special characters. The reason for that was that \hyperref[sec:wordnik]{WordNik} could not handle theses cases properly. One could try to integrate other content providers to overcome these limitations. 

At the time of writing this thesis we were unable to integrate OWL-Verbalizer, an open source tool aimed at producing texts from generic OWL ontologies, because it was originally written in SWI-Prolog but our platform runs on the Java Virtual Machine~(JVM). Even though the tool authors are aware of this problem\footnote{\url{https://github.com/Kaljurand/owl-verbalizer/issues/13} accessed 2019/02/08}, the lack of compatibility with other programming languages still remains. Future research could possibly take a closer look at the Java~Interface~to~Prolog\footnote{\url{http://www.swi-prolog.org/packages/jpl/} accessed 2018/05/11} which offers a bidirectional interface between Java and Prolog that can be used to embed Prolog in Java as well as for embedding Java in Prolog. 

Researchers could also investigate more in the direction of making our approaches more generic. This holds in particular for the Ontology~based~Approach because it only takes subsumption relations into account. A huge improvement would be to also include object properties, however, major changes would be required because the algorithm needs to consider besides concepts also individuals. 

To achieve the goal of a general purpose solution for ontology validation, our methods should be extended to cover other tasks as well.
The \hyperref[sec:ucomp_protege_plugin]{uComp~Protege~Plugin} could serve as a starting point here. Besides verification of domain relevance, other tasks include verification of relation correctness, specification of relation type and verification of domain and range. The challenging part here is to adapt the workflow of these tasks in such a way that the Crowdsourcing tasks include contextual information. 

The last open topic relates to the big picture presented in the beginning of this chapter where our contributions provide just a small portion of work that is needed towards a fully fledged ontology learning framework. In that vein, future studies need to evaluate which of the approaches of ontology validation works best~(e.g. automated or manual ones) and possibly provide a hybrid approach which combines several of these approaches.



