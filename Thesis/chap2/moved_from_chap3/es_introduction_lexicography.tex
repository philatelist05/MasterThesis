\paragraph{Short Introduction into Lexicography}
Dictionaries, as most humans are familiar with, contain the \textit{lexicon} of a language, that is a collection of words or phrases, called the lexemes. 
They are designed for word lookups and facilitate finding information about spelling, meaning and usage. Historically, written dictionaries were constrained by the alphabetised format and once printed, never updated. On the other hand, online dictionaries facilitate accessing the information on more than one path, semantics being one of them. Similarly, keeping the word up-to-date as well as adding new ones is as simple as updating or adding new records to the database, compared to printing and distributing new copies of their written pedants. 

Lexicography is defined as the science about~\enquote{the theory and practice of dictionaries, that is, dictionaries, encyclopaedias, lexica, glossaries, vocabularies, terminological knowledge bases, and other information tools covering areas of knowledge and its corresponding language.}~\cite{fuertes2017}
Among other definitions, they share the understanding that lexicography is an interdisciplinary science with characteristics of linguistic science, information science and others. 

When it comes to classifying electronic dictionaries, it is obvious that they are more than just machine-readable copies of their printed counterparts, but rather comprehensive software conglomerates only limited by technological, economical and/or practical constraints. The key defining elements of electronic dictionaries are 
\begin{inparaenum}[i)]
		\item the access principle,
		\item the construction principle and
		\item the distinction between information databases and information tools
\end{inparaenum}~\cite{fuertes2011}.
Whereas the \textit{access principle} deals with providing quick and easy access to the needed information, the \textit{construction principle} differentiates dictionaries based on the technical and economical aspects when building the dictionary. Special attention should be paid to the division between information databases and information tools, that is, separating user concerns~(searching and viewing the results) from the underlying data organisation. 

Depending on the usage scenarios, several types of dictionaries have emerged over the years. A recent study~\cite{mueller_spitzer2013} evaluated, how dictionaries are being used or, in other words, the external conditions or situations in which a dictionary consultation is embedded.  
In their research, participants with professional and academic background were asked about the circumstances of dictionary usage. 
Analyses revealed that responses can be partitioned into the following \textit{groups of usage} situations: 
\begin{enumerate}
	\item Text Production
	\item Text Reception
	\item Text Translation
\end{enumerate}

\textit{Text Production} is the action of writing texts in professional or personal contexts. Dictionary content vary in many ways, depending on writer's profession and text category. For example, passive speech using domain specific vocabularies are dominant in academic literature over active speech and simple language constructs in short-term, informal texts~\cite{o2010routledge}~(e.g. emails, tweets, Facebook posts, \ldots).

\textit{Text Reception} is the process or theory that emphasises the meaning and interpretation of existing literature. Readers focus on writing reviews and understanding texts. The user's motivation of dictionary interaction is finding the word definition as well as looking for samples~(e.g. example sentences), synonyms/antonyms and links to external information. 

\textit{Text Translation} is the process of converting text from a source language to a target language by preserving its meaning. Unfortunately there is not always a one-to-one mapping possible because of lacking equivalent language constructs in the target language. Old-fashioned dictionaries containing translations for specific words do not help either because words are lacking context. With the advent of online dictionaries, new opportunities for translators appeared such as spelling correction and links to synonyms/antonyms. Not only professional translators responded that dictionaries are crucial for their business, students and teachers consult dictionaries to improve their vocabulary, look up pronunciation or simply find the contexts of word usage. 

