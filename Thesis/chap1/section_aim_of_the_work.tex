\section{Aim of the Work}
When investigating the use of Context in Crowdsourcing tasks the first question that arise is, what is actually
meant by the term \guillemotright Context\guillemotleft~. Besides referring to the need of Context to improve the understanding
of Crowdsourcing tasks~\cite{sarasua2015crowdsourcing}, not much literature exist yet that exclusively targets this topic. An overview
of existing work as well as a conclusive definition of Context is given in~\hyperref[sec:the_use_of_context_in_crowdsourcing]{Section~\ref*{sec:the_use_of_context_in_crowdsourcing}}.

Hence, the following research questions that target the need of Context in Crowdsourcing tasks are addressed in this work:

\textbf{RQ-I} \emph{Does the crowd perform better on context enriched Crowdsourcing tasks?}

The basic question that motivates the research of this thesis is whether the performance of crowd workers could be improved if Context were added
to Crowdsourcing tasks. Researchers have already stated this hypothesis~\cite{sarasua2015crowdsourcing} not much research exists that relates to this topic. 

In order to give a detailed answer to this research question, a conclusive definition of \guillemotright Context\guillemotleft~ needs to be stated. Unfortunately no such definition exist yet, while some work was already done that use Context, either implicitly or explicitly. Whereas the first part of the previous sentence is addressed in~\hyperref[sec:context_in_crowdsourcing_tasks_context]{Section~\ref*{sec:context_in_crowdsourcing_tasks_context}} by giving a conclusive definition of \guillemotright Context\guillemotleft~,\hyperref[sec:context_in_crowdsourcing_tasks_approaches]{Section~\ref*{sec:context_in_crowdsourcing_tasks_approaches}} examines the second part of the sentence. 



\textbf{RQ-II} \emph{What methods can be applied that generate Context?}

 
\textbf{RQ-III} \emph{To what extent is it possible to transfer the investigated methods to different datasets?}


\textbf{RQ-IV} \emph{Which of the proposed methods work best? What are potential shortcomings and why?}


