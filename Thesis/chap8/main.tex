\chapter{Discussion \& Conclusion}\label{chap:discussion_and_conclusion}
% Describe briefly what this thesis is about
% Relate to chapter2 (e.g. related work) - context and approaches in particular
% Desribe then what this thesis tries to solve (e.g. briefly describe each of the 3 methods )

In the remainder of this chapter each research question is revisited by taken together all the results from the experimental evaluation~(see \hyperref[chap:results]{Chapter~\ref*{chap:results}}) and drawing final conclusions that point future research in new directions. 

The main research question examined in this thesis was:
 
\textbf{RQ-I} \emph{Does the crowd perform better on context enriched Crowdsourcing tasks?}

Answering this question might seem difficult at first because measuring the crowd's performance depends on the metrics of measurement as well as the concrete evaluation settings. However, all proposed methods performed better with regard to F-Measure than omitting Context. Indeed all our experiments showed that in each dataset the number of correct classifications with added Context is considerably higher. 

Clearly, the most important performance metric is F-Measure because by combining Precision and Recall it strengthens the benefits as well as weakens the shortcomings of both metrics. We observed that the lead of the methods with Context is slightly lower when measured by recall compared to precision. In other words, crowd workers tend to rather decline relevant concepts. Considering that our approach~(e.g. ontology validation) is embedded in a bigger ontology learning process, this seems unproblematic because domain experts and ontology engineers rather prefer deleting a few concepts rather than missing some important ones~\cite{sabou2006}. 

Based on our experiments performed on three datasets including tennis, climate change and finance we proposed a viable solution that adds Context to Crowdsourcing tasks and improves the results of the ontology validation process. It has already been mentioned~\cite{mortensen2015, mortensen2016, wohlgenannt2016} that crowd-based ontology validation is a good alternative to manual validation, especially in situations where an expert is unavailable, budget is limited or the ontology is just too large. 

\textbf{RQ-II} \emph{What methods can be applied that generate Context?}

We measured the performance of the crowd using three methods that generate concept descriptions either requiring manual intervention or being fully automated. Our proposed approaches were discussed in detail in~\hyperref[chap:context_enrichment_methods]{Chapter~\ref*{chap:context_enrichment_methods}}). 

The Ontology based Approach~(see \hyperref[sec:enrichment_ontology_approach]{Section~\ref*{sec:enrichment_ontology_approach}})) processes hierarchical relations that are encoded within the ontology. The biggest advantage of this approach is that it works without any external dependencies in a fully automated manner. This algorithmic approach is recommended for ontologies containing a large number of concepts that are connected by subsumption relations. A potential pitfall of this method is that the full potential of ACE~(Attempto Controlled English) could not be leveraged because we identified certain obstacles that hinder the integration of OWL~Verbalizer, a tool that converts an ontology into a set of ACE sentences. Consequently, our algorithm generates the text by simple string replacement, not taking the word category~(e.g. singular or plural) into account. 

The second approach~(Metadata based Approach --- see \hyperref[sec:enrichment_metaData_approach]{Section~\ref*{sec:enrichment_metaData_approach}})
is based on metadata that is encoded within the ontology. In contrast to the other methods this approach requires some manual work. As a precondition 
the metadata needs to be added by experts in a standardised format which is then provided as Context in Crowdsourcing tasks. Because the additional costs of manual preprocessing might not outweigh the benefits of high quality concept descriptions, it makes sense to preferable use this approach in very specialised areas such as in biomedical domains where ontologies are typically well documented and already contain explanations.

The idea of the Dictionary based Approach~(see \hyperref[sec:enrichment_dictionary_approach]{Section~\ref*{sec:enrichment_dictionary_approach}}))
is that starting from a concept name, descriptions are built from consulting an online dictionary.


\textbf{RQ-III} \emph{To what extent is it possible to transfer the investigated methods to different datasets?}

\textbf{RQ-IV} \emph{Which of the proposed methods work best? What are potential shortcomings and why?} 

