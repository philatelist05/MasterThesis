\chapter{Discussion \& Conclusion}\label{chap:discussion_and_conclusion}
 
The main research question examined in this thesis was:
 
\textbf{RQ-I} \emph{Does the crowd perform better on context enriched Crowdsourcing tasks?}

Answering this question might seem difficult at first because measuring the crowd's performance depends on the metrics of measurement as well as the concrete evaluation settings. However, all proposed methods performed better with regard to F-measure than omitting Context. Indeed all our experiments showed that in each dataset the number of correct classifications with added Context is considerably higher. 

Clearly, the most important performance metric is F-measure because by combining precision and recall it strengthens the benefits as well as weakens the shortcomings of both metrics. We observed that the lead of the methods with Context is slightly lower when measured by recall compared to precision. In other words, crowd workers tend to rather decline relevant concepts. Considering that our approach~(e.g. ontology validation) is embedded in a bigger ontology learning process, this seems unproblematic because domain experts and ontology engineers rather prefer deleting a few concepts rather than missing some important ones~\cite{sabou2006}. 

Based on our experiments performed on three datasets including tennis, climate change and finance we proposed a viable solution that adds Context to Crowdsourcing tasks and improves the results of the ontology validation process. It has already been mentioned~\cite{mortensen2015, mortensen2016, wohlgenannt2016} that crowd-based ontology validation is a good alternative to manual validation, especially in situations where an expert is unavailable, budget is limited or the ontology is just too large. 

\textbf{RQ-II} \emph{What methods can be applied that generate Context?}

\textbf{RQ-III} \emph{To what extent is it possible to transfer the investigated methods to different datasets?}

\textbf{RQ-IV} \emph{Which of the proposed methods work best? What are potential shortcomings and why?} 

