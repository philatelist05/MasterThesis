\section{Embedded Context}\label{sec:embedded_context}
Over the years ontologies were used in many domain contexts, including general-purpose as well as highly specialised ones. Obviously, what separates good ontologies from poor ones is how well they are documented~\cite{daquin2012}. Studies~\cite{dutta2017} analysed various approaches of embedding meta-data in ontologies. The outcome was that there is no standard way to describe and document ontologies, albeit a few vocabularies that describe semantic meta-data exist. 

This section starts by introducing annotation properties which are defined as part of the \hyperref[sec:OWL_annotation_properties]{Web~Ontology~Language~(OWL)\footnote{\url{https://www.w3.org/OWL/} accessed 2018/18/12}}. We used annotations to encode the context descriptions. Next, an overview of the \hyperref[sec:dublin_core_metadata_vocabulary]{Dublin~Core~Metadata~Set} is given as some parts were used for the definition of context properties. Then, in the remainder of this section our \hyperref[sec:enrichment_metaData_approach]{Meta-Data~based~Approach} is discussed.

\subsection{OWL Annotation Properties}\label{sec:OWL_annotation_properties}

\subsection{Dublin Core~(DC)}\label{sec:dublin_core_metadata_vocabulary} Being one of the most prominent vocabulary in describing semantic meta-data, published and maintained by the Dublin Core Metadata Initiative~(DCMI), it originally contained 15 meta-data terms\footnote{\url{http://www.dublincore.org/documents/dces/} accessed 2018/05/20},  designed to annotate resources with simple, textual information. Since its first launch, the project have gained popularity, including more than 127 terms\footnote{\url{http://www.dublincore.org/documents/dcmi-terms/} accessed 2018/05/20}. The initial set of terms is listed in~\hyperref[app:dc_terms]{Appendix~\ref*{app:dc_terms}}. 

To maximise interoperability in heterogeneous environments, an RDF-Schema with DCMI-Metadata\footnote{\url{http://dublincore.org/schemas/rdfs/} accessed 2018/05/20} elements was created, in which each entity is identified by a Uniform Resource Identifier~(URI) starting with the prefix \emph{http://purl.org}. A broader discussion on the use of meta-data in general is given in~\cite{nilsson2010}.  

\subsection{Meta-Data based Approach}\label{sec:enrichment_metaData_approach}
Given the high number on ontology meta-data formats from above, \hyperref[alg:embedded_enrichment]{Algorithm~\ref*{alg:embedded_enrichment}} shows the pseudocode to create concept descriptions extracted from embedded meta-data. In addition to the notation used in the previous section we define $\Phi(C) \coloneqq \{m_1, m_2, \ldots, m_i \}$ where $m_i$ is the $i'th$ meta-data element embedded in concept $C$ and $T$ is the description of some meta-data element.

\begin{algorithm}
	\caption{Context Enrichment based on embedded meta-data}\label{alg:embedded_enrichment}
	\begin{algorithmic}[1]
		\Procedure{Generate Description}{}\newline
			\textbf{Input:} A concept $C$ with embedded meta-data $\{m_1, m_2, \ldots, m_i \}$\newline
			\textbf{Output:} A description $T$ of $C's$ meta-data elements\newline
			\State{$T=\{\}$}
			\For {$ m_k \in \Phi(C) $}
				\State $T=T$ $\cup$ $m_k$
			\EndFor
		\EndProcedure
	\end{algorithmic}
\end{algorithm}

While the actual enrichment is straightforward, it collects all descriptions for a determined concept, the details of extracting the meta-data from annotation properties is omitted here because it highly depends on the chosen meta-data encoding.
As we decided to encode the meta-data in annotation properties, the extraction process works by selecting the related annotation properties for a specified concept. 
