\section{Summary}\label{sec:approaches_summary}
In this chapter we investigated three approaches that generate descriptions for selected concepts. The motivation was 
the limitations imposed by the existing platform for crowd-sourced ontology validation~(uComp Protege Plugin). All our approaches
were integrated as an extension of the platform which facilitates comparability of crowd worker performance described extensively in
the~\hyperref[chap:results]{Results Chapter}. 

This first method~\hyperref[sec:enrichment_ontology_approach]{(\emph{Ontology based Approach})} uses the ontology graph, more precisely
subsumption relations, to generate the concept descriptions. For the second method~\hyperref[sec:enrichment_metaData_approach]{\emph{(Metadata
based Approach)}} additional metadata encoded within the ontology is processed. Therefore, some parts of the Dublin~Core~Metadata~Set were used
which define a standard set of OWL annotations and extend them with common semantics. The basic principle of the third 
method~\hyperref[sec:enrichment_dictionary_approach]{(\emph{Dictionary based Approach})} was the retrieval of definitions from external sources. For that, WordNik, a freely available online dictionary for the English language which aggregates results from across the Web, was consulted to find example sentences for some selected concepts.

