\chapter{Summary \& Future Work}\label{chap:summary_and_future_work}


%SUMMARY%
%1.st paragraph: briefly describe the motivation and approach									DONE
%2.nd paragraph: briefly describe proposed methods												DONE
%3.rd paragraph: briefly describe the results

%FUTURE WORK%
%4.th ff paragraphs: describe big pricture
	% Proposed solution is embedded into bigger context (e.g. ontology learning solution)
	% Provides only a tiny poprtion

%paragraph: add transition text to open questions --- future tasks
%paragraph: for each open task add a new paragraph here

In this thesis we investigated whether contextual information in Crowdsourcing tasks helped to achieve better results for performing ontology validation using Crowdsourcing techniques. Crowdsourcing is a technique of distributing small tasks to a typically large group of human workers. It offers a cost effective method of solving tasks which are traditionally hard for machines but easily solvable by humans. Contextual information is any kind of of additional information that is supplied with a crowdsourcing task that improves the understanding of the task goals.

We presented three novel methods that enrich Crowdsourcing tasks with contextual information to validate the relevance of concepts for a particular domain of interest. Whereas the Ontology~based~Approach processes hierarchical relations, the Metadata~based~Approach generates descriptions based on annotations that were encoded within the ontology, the idea of the Dictionary~based~Approach is to form the explanations from example sentences by conducting the online dictionary \hyperref[sec:wordnik]{WordNik}.

