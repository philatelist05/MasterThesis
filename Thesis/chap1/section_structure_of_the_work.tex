\section{Structure of the Work}
The structure of this thesis is as follows:

\hyperref[chap:state_of_the_art]{Chapter~\ref*{chap:state_of_the_art}} introduces the concepts that are used throughout this thesis. It includes
\begin{inparaenum}[i)]
		\item a brief introduction into Crowdsourcing in~\hyperref[sec:state_of_the_art_crowdsourcing]{Section~\ref*{sec:state_of_the_art_crowdsourcing}},
		\item a discussion on the interplay between Crowdsourcing and the Semantic Web in~\hyperref[sec:state_of_the_art_crowdsourcing_and_the_semantic_web]{Section~\ref*{sec:state_of_the_art_crowdsourcing_and_the_semantic_web}},
		\item a detailed presentation of the uComp Protege Plugin which served as the baseline of this work~(\hyperref[sec:ucomp_protege_plugin]{Section~\ref*{sec:ucomp_protege_plugin}}), and 
		\item a review of existing literature that addressed the use of Context in Crowdsourcing tasks which was taken together to formulate a conclusive definition of the term \guillemotright Context\guillemotleft~in \hyperref[sec:the_use_of_context_in_crowdsourcing]{Section~\ref*{sec:the_use_of_context_in_crowdsourcing}}.
\end{inparaenum}

\hyperref[chap:context_enrichment_methods]{Chapter~\ref*{chap:context_enrichment_methods}} introduces the proposed methods that enrich Crowdsourcing tasks with Context. While in~\hyperref[sec:neighboring_nodes]{Section~\ref*{sec:neighboring_nodes}} the Ontology based Approach is discussed,~\hyperref[sec:embedded_context]{Section~\ref*{sec:embedded_context}} explains the Metadata based Approach and~\hyperref[sec:external_source]{Section~\ref*{sec:external_source}} presents the Dictionary based Approach. Then,~\hyperref[chap:experimental_evaluation]{Chapter~\ref*{chap:experimental_evaluation}} is dedicated to the evaluation settings. Concretely, it presents the metrics~(\hyperref[sec:evaluation_metrics]{Section~\ref*{sec:evaluation_metrics}}) that served as performance measure and the used
datasets~(\hyperref[sec:evaluation_datasets]{Section~\ref*{sec:evaluation_datasets}}).
Next, in~\hyperref[chap:results]{Chapter~\ref*{chap:results}} the obtained results are analysed.

In~\hyperref[chap:conclusion_and_futue_work]{Chapter~\ref*{chap:conclusion_and_futue_work}}, the main topics of this thesis are summarised and the research questions are revisited. Finally, an outlook for future research topics is provided.
