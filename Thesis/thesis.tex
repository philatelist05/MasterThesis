% Copyright (C) 2014-2017 by Thomas Auzinger <thomas@auzinger.name>

\documentclass[draft,final]{vutinfth} % Remove option 'final' to obtain debug information.

% Load packages to allow in- and output of non-ASCII characters.
\usepackage{lmodern}        % Use an extension of the original Computer Modern font to minimize the use of bitmapped letters.
\usepackage[T1]{fontenc}    % Determines font encoding of the output. Font packages have to be included before this line.
\usepackage[utf8]{inputenc} % Determines encoding of the input. All input files have to use UTF8 encoding.
\usepackage{import} %Better multi-document handling

% Extended LaTeX functionality is enables by including packages with \usepackage{...}.
\usepackage{amsmath}    % Extended typesetting of mathematical expression.
\usepackage{amssymb}    % Provides a multitude of mathematical symbols.
\usepackage{amsthm}     % Provides environments for theorems, definitions, ...
\usepackage{mathtools}  % Further extensions of mathematical typesetting.
\usepackage{microtype}  % Small-scale typographic enhancements.
\usepackage{paralist} %Allows usage of inline lists
\usepackage[inline]{enumitem} % User control over the layout of lists (itemize, enumerate, description).
\usepackage{multirow}   % Allows table elements to span several rows.
\usepackage{booktabs}   % Improves the typesettings of tables.
\usepackage{subcaption} % Allows the use of subfigures and enables their referencing.
%\usepackage[ruled,linesnumbered,algochapter]{algorithm2e} % Enables the writing of pseudo code.
\usepackage{algorithm} %For algorithms
\usepackage[noend]{algpseudocode}%Allows pseudo-code like syntax in algorithms
\usepackage{tabularx, booktabs} %Better Table support
\usepackage[usenames,dvipsnames,table]{xcolor} % Allows the definition and use of colors. This package has to be included before tikz.
\usepackage{nag}       % Issues warnings when best practices in writing LaTeX documents are violated.
\usepackage{todonotes} % Provides tooltip-like todo notes.
\usepackage{hyperref}  % Enables cross linking in the electronic document version. This package has to be included second to last.
\usepackage[acronym,toc]{glossaries} % Enables the generation of glossaries and lists fo acronyms. This package has to be included last.
\usepackage[babel=true]{csquotes} %Just more quote handling
\usepackage{listings} % General purpose package for e.g. code listings
\usepackage{rotating} % Used to handle landscape orientation of large graphics



% Define convenience functions to use the author name and the thesis title in the PDF document properties.
\newcommand{\authorname}{Stefan Gamerith} % The author name without titles.
\newcommand{\thesistitle}{Context Enrichment of Crowdsourcing Tasks for Ontology Validation} % The title of the thesis. The English version should be used, if it exists.

% Make listings look exactily like verbatim environment
\lstset{
  basicstyle=\tiny\ttfamily,
  columns=fixed,
  fontadjust=true,
  keepspaces=true,
  basewidth=0.5em
}

%Evenly distribute Y colum types%
\newcolumntype{Y}{>{\centering\arraybackslash}X}

%Define hyphen in math-mode
\mathchardef\mhyphen="2D

% Set PDF document properties
\hypersetup{
    pdfpagelayout   = TwoPageRight,           % How the document is shown in PDF viewers (optional).
    linkbordercolor = {Melon},                % The color of the borders of boxes around crosslinks (optional).
    pdfauthor       = {\authorname},          % The author's name in the document properties (optional).
    pdftitle        = {\thesistitle},         % The document's title in the document properties (optional).
    pdfsubject      = {Subject},              % The document's subject in the document properties (optional).
    pdfkeywords     = {a, list, of, keywords} % The document's keywords in the document properties (optional).
}

%% THEOREM, DEFINITION, EXAMPLE handling  %%
\theoremstyle{plain}
\newtheorem{thm}{Theorem}[chapter] % reset theorem numbering for each chapter

\theoremstyle{definition}
\newtheorem{defn}[thm]{Definition} % definition numbers are dependent on theorem numbers
\newtheorem{exmp}[thm]{Example} % same for example numbers


\setpnumwidth{2.5em}        % Avoid overfull hboxes in the table of contents (see memoir manual).
\setsecnumdepth{subsection} % Enumerate subsections.

\nonzeroparskip             % Create space between paragraphs (optional).
\setlength{\parindent}{0pt} % Remove paragraph identation (optional).

\makeindex      % Use an optional index.
\makeglossaries % Use an optional glossary.
%\glstocfalse   % Remove the glossaries from the table of contents.

% Set persons with 4 arguments:
%  {title before name}{name}{title after name}{gender}
%  where both titles are optional (i.e. can be given as empty brackets {}).
\setauthor{}{\authorname}{BSc.}{male}
\setadvisor{}{Reka Marta Sabou}{MSc., PhD}{female}

% For bachelor and master theses:
%\setfirstassistant{Pretitle}{Forename Surname}{Posttitle}{male}
%\setsecondassistant{Pretitle}{Forename Surname}{Posttitle}{male}
%\setthirdassistant{Pretitle}{Forename Surname}{Posttitle}{male}

% For dissertations:
%\setfirstreviewer{Pretitle}{Forename Surname}{Posttitle}{male}
%\setsecondreviewer{Pretitle}{Forename Surname}{Posttitle}{male}

% For dissertations at the PhD School and optionally for dissertations:
% \setsecondadvisor{Pretitle}{Forename Surname}{Posttitle}{male} % Comment to remove.

% Required data.
\setaddress{Linzerstrasse 429/4215, 1140 Wien}
\setregnumber{0925081}
\setdate{01}{01}{2001} % Set date with 3 arguments: {day}{month}{year}.
\settitle{\thesistitle}{\thesistitle} % Sets English and German version of the title (both can be English or German). If your title contains commas, enclose it with additional curvy brackets (i.e., {{your title}}) or define it as a macro as done with \thesistitle.
%\setsubtitle{Optional Subtitle of the Thesis}{Optionaler Untertitel der Arbeit} % Sets English and German version of the subtitle (both can be English or German).

% Select the thesis type: bachelor / master / doctor / phd-school.
% Bachelor:
%\setthesis{bachelor}
%
% Master:
\setthesis{master}
\setmasterdegree{dipl.} % dipl. / rer.nat. / rer.soc.oec. / master
%
% Doctor:
%\setthesis{doctor}
%\setdoctordegree{rer.soc.oec.}% rer.nat. / techn. / rer.soc.oec.
%
% Doctor at the PhD School
%\setthesis{phd-school} % Deactivate non-English title pages (see below)

% For bachelor and master:
\setcurriculum{Software Engineering / Internet Computing}{Software Engineering / Internet Computing} % Sets the English and German name of the curriculum.

% For dissertations at the PhD School:
%\setfirstreviewerdata{Affiliation, Country}
%\setsecondreviewerdata{Affiliation, Country}


\begin{document}

\frontmatter % Switches to roman numbering.
% The structure of the thesis has to conform to
%  http://www.informatik.tuwien.ac.at/dekanat

\addtitlepage{naustrian} % German title page (not for dissertations at the PhD School).
\addtitlepage{english} % English title page.
\addstatementpage

%%%%%%%%%%%%%%%%%%%%%%%%%%%%%%%%%%%%%%%%%%%%%%%%%%%%%%%%%%%%%%FRONT MATTER%%%%%%%%%%%%%%%%%%%%%%%%%%%%%%%%%%%%%%%%%%%%%%%%%%%%%%%%%%%%%%%%%%%%%%%%%%

% DANKSAGUNG %
\begin{danksagung*}
\todo{Ihr Text hier.}
\end{danksagung*}

% ACKNOWLEDGEMENTS %
\begin{acknowledgements*}
First and foremost I am grateful for the support from my family, both financially and mentally. Special thanks goes to my mother Christa and father Willibald for always being there for me and giving me the strength and stability during the writing period, especially in those situations where I would otherwise give up.

Next, I want to thank my advisor Stefan Biffl and his assistance Marta Sabou for the opportunity to write this diploma thesis. As a side note, the inspiration for this thesis was actually from a seminar course I took some time ago with some of my colleagues. That was the time where I met Stefan Biffl. He showed me all the details of writing scientific texts. Most importantly, I really liked his structured and organised way of thinking and working. 

Last, I want to thank all contributors from Australia, the United Kingdom and the United States of America who took part in our Crowdsourcing experiment.

\end{acknowledgements*}

% KURZFASSUNG %
\begin{kurzfassung}
	%Ein wichtiger Bestandteil des Semantik Web Lebenszyklus ist die Überprüfung der Ontologie Relevanz. Dies ist insbesondere der Fall bei erlernten 
	%Ontologien, welche von Natur aus Fehler enthalten. Obwohl viele davon von Algorithmen gelöst werden können, ist dies mitunter bei
	%komplexen Problemstellungen schwierig. Crowdsourcing stellt eine kosteneffiziente Alternative dar die diese Problemstellungen mit menschlicher
	%Kraft löst. Dennoch ist die Performance jener Ansätze die Crowdsourcing mit Ontologie Validierung verknüpft wenig zufriedenstellend. 
	%
	%Ein vielversprechender Ansatz dieses Problem zu lösen wäre, wenn Crowdsourcing Aufgaben zusätzlich kontextbezogene Informationen zur besseren
	%Verständlichkeit enthielten. Dieser Kontext hätte nicht nur einen positiven Einfluss auf die Performance der Teilnehmer, sondern würde auch 
	%zur besseren Qualität der Ergebnisse beitragen.
	%
	%Obwohl gewisse Fortschritte in diesem Bereich erst kürzlich erzielt wurden, befasste sich keine der Publikationen mit dem Thema Kontext
	%im Zusammenhang mit Crowdsourcing. In dieser Diplomarbeit präsentieren wir 3 neuartige Methoden die Kontext für Crowdsourcing Aufgaben herstellen 
	%um die Relevanz von Konzepten für eine bestimmte Domäne zu überprüfen. Während die auf Ontologien basierende Methode hierarchische Relationen
	%verarbeitet, generiert die auf Metadaten basierende Methode Beschreibungen welche auf Annotationen beruhen. Durch Suchabfrage über die Platform
	%WordNik	werden Beispielsätze geformt, welche die Basis für die dritte Methode bilden. 
	%
	%Zur Auswertung der Ergebnisse integrierten wir alle 3 Methoden in das bestehende uComp Protege Plugin, welches die Eingliederung von Crowdsourcing 
	%Aufgaben zur Validierung von Ontologien innerhalb des Ontologie Editors Protege ermöglicht. Die Methoden wurden an 3 Ontologien der Bereiche
	%Klimawandel, Tennis und Finanzen getestet. Die Metriken Precision, Recall und F-Measure wurden für jeden Datensatz berechnet um Rückschlüsse über 
	%die Performance der getesteten Methoden ziehen zu können. Die Auswertungen ergaben, dass die auf Metadaten basierende Methode die besten
	%Ergebnisse lieferte. Die anderen Methoden hatten Probleme in bestimmten Situationen, beispielsweise konnte die Verzeichnis basierende Methode 
	%bei mehrdeutigen Konzepten nicht überzeugen. Weiters hatte die Ontologie basierende Methode Schwierigkeiten bei Ontologien welche nur aus wenigen
	%Subklassen Beziehungen bestanden. Alle 3 Methoden lieferten Ergebnisse von höchster Qualität (F-Measure größer 80\%) wodurch die
	%Performance der Teilnehmer durch das Hinzufügen von kontextbezogener Information gesteigert werden konnte. 
	
	Ein wichtiger Teil des Semantik Web Lebenszyklus ist die Ontologie Validierung,
	insbesondere bei erlernten Ontologien, die von Natur aus Fehler enthalten.
	Obwohl mittlerweile viele dieser Fehler von Algorithmen erkannt werden,
	ist dies mitunter bei komplexen Problemstellungen schwierig. Crowdsourcing stellt eine
	kosteneffiziente Alternative dar, die diese Aufgaben an eine Gruppe freiwilliger User (Crowd)
	auslagert. Dennoch gibt es bei der Ontologie Validierung mittels Crowdsourcing Verbesserungsbedarf.

	Ein Lösungsansatz wäre die Zugabe kontextbezogener Informationen zu Crowdsourcing Aufgaben.
	Dies hätte mitunter einen positiven Einfluss auf das Validierungsergebnis.

	Obwohl Fortschritte in diesem Bereich erzielt wurden, gibt es noch wenig Literatur
	zu diesem Thema. In dieser Diplomarbeit stellen wir 3 Methoden vor, die Kontext 
	generieren um die Relevanz von Konzepten innerhalb einer
	Domäne zu überprüfen. Während der Ontology-based-Approach hierarchische
	Relationen verarbeitet, basiert der Metadata-based-Approach auf Annotationen.
	Als Basis für die letzte Methode (Dictionary-based-Approach) dienen Beispielsätze des Online
	Wörterbuchs WordNik.

	Alle 3 Methoden wurden als Erweiterung des uComp Protege Plugin konzipiert, 
	ein Plugin für den Ontologie Editor Protege, das die Validierung von Ontologien mittels Crowdsourcing
	ermöglicht. Im Rahmen von 3 Experimenten mit Datensätzen aus den Bereichen Klimawandel, Tennis und Finanzen
	wurden alle 3 Methoden getestet. Die Metriken Precision, Recall und F-Measure wurden für jeden Datensatz berechnet
	um Rückschlüsse über die Performance der getesteten Methoden ziehen zu können. Der Metadata-based-Approach
	lieferte die besten Validierungsergebnisse. Anhand der guten bis sehr guten Ergebnisse aller 3 Methoden (F-Measure
	größer 80\%) wurde gezeigt, dass die Qualität der Validierung durch das Hinzufügen kontextbezogener Information
	gesteigert werden konnte.

\end{kurzfassung}

% ABSTRACT %
\begin{abstract}
	% ROADMAP for the Abstract (1 page):
	% 1st paragraph: Problem statement
	% 2ns paragraph: Motivation for writing the thesis
	% 3rd paragraph: The contributions of the thesis (e.g. the proposed approaches)
	% 4th paragraph: Evaluation (metrics, used datasets, RESULTS --- outcome of the evaluation part)
	
\end{abstract}

%%%%%%%%%%%%%%%%%%%%%%%%%%%%%%%%%%%%%%%%%%%%%%%%%%%%%%%%%%%%%%%%%%%END%%%%%%%%%%%%%%%%%%%%%%%%%%%%%%%%%%%%%%%%%%%%%%%%%%%%%%%%%%%%%%%%%%%%%%%%%%%%%%%


% Select the language of the thesis, e.g., english or naustrian.
\selectlanguage{english}

% Add a table of contents (toc).
\tableofcontents % Starred version, i.e., \tableofcontents*, removes the self-entry.

% Switch to arabic numbering and start the enumeration of chapters in the table of content.
\mainmatter



%%%%%%%%%%%%%%%%%%%%%%%%%%%%%%%%%%%%%%%%%%%%%%%%%%%%%%%%%%%%%%%%%%%%%%%%%%%%%%%%%%%%%%%%%%%%%%%%%%%%%%%%%%%%%%%%%%%%%%%%%%%%%%%%%%%%%%%%%%%%%%%%%%%%

% INTRODUCTION %
\subimport*{chap1/}{main.tex}

% STATE OF THE ART %
\subimport*{chap2/}{main.tex}

% CONTEXT ENRICHMENT METHODS %
\subimport*{chap3/}{main.tex}

% EXPERIMENTAL EVALUATION %
\subimport*{chap4/}{main.tex}

% CONCEPTUAL ARCHITECTURE - (Optional) %
% \subimport*{chap5/}{main.tex}

% EVALUATION OF THE CODE - (Optional) %
% \subimport*{chap6/}{main.tex} %

% RESULTS %
\subimport*{chap7/}{main.tex}

% DISCUSSION & CONCLUSION %
\subimport*{chap8/}{main.tex}

% SUMMARY & FUTURE WORK %
\subimport*{chap9/}{main.tex}

% APPENDIX %
\subimport*{appendix/}{main.tex}

%%%%%%%%%%%%%%%%%%%%%%%%%%%%%%%%%%%%%%%%%%%%%%%%%%%%%%%%%%%%%%%%%%%%%%%%%%%%%%%%%%%%%%%%%%%%%%%%%%%%%%%%%%%%%%%%%%%%%%%%%%%%%%%%%%%%%%%%%%%%%%%%%%%%


\backmatter

% Use an optional list of figures.
\listoffigures % Starred version, i.e., \listoffigures*, removes the toc entry.

% Use an optional list of tables.
\cleardoublepage % Start list of tables on the next empty right hand page.
\listoftables % Starred version, i.e., \listoftables*, removes the toc entry.

% Use an optional list of alogrithms.
\listofalgorithms
\addcontentsline{toc}{chapter}{List of Algorithms}

% Add an index.
%\printindex

% Add a glossary.
%\printglossaries

% Add a bibliography.
\bibliographystyle{alpha}
\bibliography{literature}

\end{document}