\chapter{Context Enrichment Methods}\label{chap:context_enrichment_methods}
In this chapter we describe our approaches of generating context descriptions in detail. All described approaches do not rely on some pre-defined settings or data, instead they are applicable to general purpose ontologies as well as specialised ones of arbitrary size.

In this chapter, we present three context enrichment methods for ontology validation. While the first one~(\hyperref[sec:enrichment_ontology_approach]{\textbf{Ontology~based~Approach}}), introduced in \hyperref[sec:neighboring_nodes]{Section~\ref*{sec:neighboring_nodes}}, takes neighbouring nodes~(i.e. subclass relations) into account, the second one~(\hyperref[sec:enrichment_metaData_approach]{\textbf{Metadata based Approach}}), discussed in \hyperref[sec:embedded_context]{Section~\ref*{sec:embedded_context}}, is based on embedded context and the last one~(\hyperref[sec:enrichment_dictionary_approach]{\textbf{Dictionary based Approach}}), explained in \hyperref[sec:external_source]{Section~\ref*{sec:external_source}}, uses external sources as input for enrichment. 

%%%%%%%%%%%%%%%%%%%%%%%%%%%%%%%%%%%%%%%%%%%%%%%%%%%%%%%%%%%%%%%%%%%%%%%%%%%%%%%%%%%%%%%%%%%%%%%%%%%%%%%%%%%%%%%%%%%%%%%%%%%%%%%%%%%%%%%%%%%%%%%%%%%%

% SECTION: INTRODUCTION %
\section{Introduction}\label{sec:approaches_introduction}
Previous experiments using the uComp~Protege~Plugin~\cite{wohlgenannt2016} where this thesis builds up on, had successfully applied crowdsourcing techniques on ontology validation. They conclude that it leads to data quality comparable to that of manually performed validation done by ontology engineers while reducing the overall costs. However, there was still potential to improve the worker performance in terms of speed and quality. 
Studies~\cite{mortensen2013} confirmed this statement concluding that the best worker performance is achieved
\enquote{with questions formulated in the most basic form, a domain-specific qualification, and concept definitions for context}.

Based on the existing platform for creating crowdsourcing jobs within Protege, we fill this gap by investigating different approaches of context creation. 

% SECTION: NEIGHBOURING NODES %
\section{Neighbouring Nodes}\label{sec:neighboring_nodes}
This section starts with a conceptual overview of how neighbouring nodes in an ontology graph are used to generate contextual information. Then, existing approaches on generating textual definitions are examined. In the literature~\cite{soton265735} this task is also known under the term \textit{ontology verbalisation}. Even though there exists \textit{OWL Verbalizer}\footnote{\url{http://mcs.open.ac.uk/nlg/SWAT/Verbaliser.html} accessed 2018/04/30}, a tool which already transforms generic ontologies into English sentences, we could not integrate it into the context enrichment process because 
\begin{inparaenum}[a)]
		\item it was designed as a standalone tool written in SWI-Prolog\footnote{\url{http://www.swi-prolog.org/} accessed 2018/04/30} and
		\item it only accepts the whole ontology as input
\end{inparaenum}.
However, in our approach we adopted some of the rules proposed by OWL~Verbalizer and integrated them in the enrichment process.

To illustrate the concept of neighbouring nodes and how they relate to the context enrichment approach explained later, a simple ontology graph describing the teacher/pupil domain is given in~\hyperref[fig:simple_owl_graph]{Figure~\ref*{fig:simple_owl_graph}}.
\begin{figure}
	 \centering
	 \includegraphics[width=\textwidth]{drawio/University_Ontology_Example}
	 \caption{Simple Ontology Graph}\label{fig:simple_owl_graph}
\end{figure}
For example, if the concept \textit{:Teacher} is taken as reference node, it makes sense to not only include the concept itself, but to also consider connected nodes~(e.g. the concept \textit{:Person} and the object property \textit{:teaches}) as well. 

As in the guidelines for conducting crowdsourcing research~\cite{sarasua2015crowdsourcing}, the authors recommended to avoid technical terms in crowdsourcing questions. In the next paragraphs we explain how \textit{ontology verbalisation} helps to achieve this goal. 

\subsection{Attempto Controlled English (ACE)}
Despite the fact that natural language is desirable for descriptions as everybody knows and understands with no extra learning effort, it conflicts in terms of expressiveness and specificity with well defined ontologies which can encode complex data and relations in domain-specific areas. To resolve this conflict, a new language variant named \textbf{ACE}~\textit{(Attempto Controlled English)}~\cite{fuchs2008} was created. ACE is a formal language, capable of expressing domain-specific knowledge with a well defined syntax, supporting formal reasoning and readable by specialists who are yet unfamiliar with formal languages and methods.

To get a better understanding of ACE\footnote{\url{https://tinyurl.com/yc3zhu9a} accessed 2018/05/05}\footnote{\url{https://tinyurl.com/ycst39jv} accessed 2018/05/05}, a short overview of its language structure is given in the next paragraphs:
 
\paragraph{Simple Sentences} A Simple Sentence derived from standard English language contains a subject, a verb and additional elements: \texttt{subject + verb + complements~[~+~adjuncts~]}. The verb relates directly or indirectly to one or more other objects~(\textit{complements}). Optionally, to add more specificity, one or more adverbs and prepositional phrases can be added~(\textit{adjuncts}). 

\paragraph{Composite Sentences} A Composite Sentence is composed of one or more Simple~Sentences, connected by \textit{coordination},
\textit{subordination}, \textit{quantification} and \textit{negation}. Whereas coordination links sentences either by the word \texttt{and} or \texttt{or}, subordination relates dependent sentences in some way~(e.g. if-then sentences). Quantification allows statements about all~(universal quantification) or certain~(existential quantification) objects of a certain domain. Last, encoding negative polarity in a sentence~(e.g. sentences containing \texttt{not} or \texttt{no}) is defined as negation. 

\paragraph{Query Sentences} Query Sentences can be divided into polar questions~(e.g. with \textit{yes/no} answer) and non-polar questions, also known as \emph{wh-questions}. In contrast to yes/no questions no pre-defined answer exist for these. Furthermore, wh-questions start with either of the following five W-words: \texttt{Who}, \texttt{What}, \texttt{When}, \texttt{Where} and \texttt{Why}. However, this definition somewhat less strict as sometimes questions starting with the word \texttt{How} are included as well.

\paragraph{Anaphoric References} If the meaning of a word or phrase is context dependent, recurring occurrences of these expressions are called \textit{Anaphoric References}. More specifically, the referring term~(\textit{anaphor}) relates to an antecedent expression. For example, given the sentence: \texttt{Tom arrived, but nobody noticed him}, the pronoun \texttt{him} relates to \texttt{Tom}. To resolve ambiguities during the processing phase, Anaphoric References are replaced by encoded references. 

\subsection{OWL Verbalizer}
OWL~Verbalizer~\cite{stevens2011}, an open source tool aimed at producing texts from generic OWL ontologies, is a good example of a successful integration of ACE. A description of the basic concepts of OWL Verbaliser is given below:

This tool, being now part of the \textbf{SWAT Tool Suite}\footnote{\url{http://mcs.open.ac.uk/nlg/SWAT/} accessed 2018/05/06}, was created to overcome the burden of maintaining ontology definitions by hand. Whereas producing high quality texts in restricted application domains is under active research, this tool produces understandable descriptions using general-purpose methods that are of moderate quality.

The high-level process of ontology verbalisation shown in \hyperref[fig:verbaliser_architecture]{Figure~\ref*{fig:verbaliser_architecture}} consists of the following stages:

\paragraph{Transcoding from OWL to Prolog} The output of this stage is a file in a convenient Prolog format which was generated from an ontology in RDF/XML format. The conversation process covered by the \textit{Transcoder}, \textit{Identifier~Selector} and \textit{Label~Selector} combines identifiers~(concepts, individuals, object~properties) and labels. In addition, ambiguous terms in identifiers are normalised. 

\paragraph{Constructing a lexicon for atomic entities} The output of this stage, covered by the component \textit{Lexicon Generator}, is a collection of lexicons, computed from the normalised Prolog terms in the previous stage. A lexical entry~(lexicon) is defined as a quadruple having the following form: \texttt{<identifier, part-of-speech, singular-form, plural-form>} To facilitate processing in later stages, normalised identifiers for concepts, individuals and object~properties are stored together with word category, singular form and plural form. The word~category, also known as part-of-speech, groups words based on similar properties in terms of syntax and grammar. Common categories are nouns, verbs and adjectives. Last, to differentiate the quantity of a phrase or word, singular and plural form of a noun are associated with each lexical entry. 
Among the storage of lexicons, the algorithm also implements other rules regarding text processing. Some simple heuristics are used for pre-processing to transform the resulting word string into better readable English sentences. 

\paragraph{Selecting the axioms relevant for describing each class} This stage and the next stage are covered by the \textit{Planner}.
This component has as input all axioms from the source ontology as well as the lexicons from the previous stage. In this stage, axioms are mapped to matching lexical entries. As matching criteria the algorithm uses the lexicon identifier and the IRI~\cite{rfc3987}. 

\paragraph{Aggregating axioms with a similar structure} This stage is optional, as it is not strictly required for text generation which is described in the next stage. However, some improvements are achieved if similar axioms are aligned. 

\paragraph{Generating sentences from axioms} The final stage is covered by the component named \textit{Realiser} which forms the central part of the
verbalisation process. English sentences are generated for each axiom using logical rules for almost every logical pattern in OWL-DL. These rules are expressed in Prolog clauses, taking the axiom and optionally the lexicon as input.

\begin{figure}
	 \centering
	 \includegraphics[width=0.5\textwidth]{drawio/Ontology_Verbaliser_Architecture}
	 \caption{Conceptual architecture of the OWL ontology verbaliser~\cite{stevens2011}}\label{fig:verbaliser_architecture}
\end{figure}

Although OWL~Verbaliser would be useful to integrate in the enrichment process, there are some major obstacles:

The first is \textbf{incompatibility on a language level}. Traditionally, software systems are written in many different programming languages, leading to the challenge of dealing with interoperability~\cite{malone2014}. While OWL~Verbaliser was written in SWI-Prolog, Protege runs on the Java Virtual Machine~(JVM), causing a conceptual mismatch in programming languages and paradigms. Moreover, interoperability between conceptually different programming languages is challenging in its own\footnote{\url{http://www.swi-prolog.org/packages/jpl/} accessed 2018/05/11} and would conflict with the goal of an easy-to-integrate solution.

Another obstacle is a \textbf{mismatch on the scope of operation}. OWL~Verbaliser was implemented as a tool assisting engineers in ontology creation, a very time consuming task. It was designed as a standalone tool, launched from the command line or deployed as a web service. On the other hand, ontology enrichment is embedded in Protege and part of the ontology validation process, operating only on small parts of the ontology. In contrast, OWL~Verbaliser takes the whole ontology as input. 

\subsection{Ontology based Approach}\label{sec:enrichment_ontology_approach}
Due to this complicating issues, we implemented a different approach using some insights from OWL~Verbaliser. The pseudocode of the overall workflow is given in \hyperref[alg:neighbourhood]{Algorithm~\ref*{alg:neighbourhood}}. The notation to describe properties and relations is based on a formal Ontology~Description Logic~(DL)~\cite{baader2003}, string manipulations were formally defined in~\cite{hopcroft1969}.

\begin{algorithm}
	\caption{Context Enrichment based on Neighbouring Nodes}\label{alg:neighbourhood}
	\begin{algorithmic}[1]
		\Procedure{Generate Description}{}\newline
			\textbf{Input:} A concept $C$\newline
			\textbf{Output:} A textual description $T$ of $C's$ neighbouring nodes based on subsumption\newline
			\State{$T=\{\}$} \label{alg:neighbourhood:text_initialisation}
			\For {$ (c,d) \in C \sqsubseteq D $}
				\State $T=T$ $\cup$ "Every " $\cup$ $name(c)$ $\cup$ " is a " $\cup$ $name(d)$
			\EndFor
			\For {$ (e,c) \in E \sqsubseteq C $}
				\State $T=T$ $\cup$ "Every " $\cup$ $name(e)$ $\cup$ " is a " $\cup$ $name(c)$
			\EndFor
		\EndProcedure
	\end{algorithmic}
\end{algorithm}

The main work is done in two for-loops, which calculate context descriptions based on subsumption~$(\sqsubseteq)$ and string concatenation~$(\cup)$. To handle the case of missing subsumption relations, the output text $T$ is initialised to an empty string~(\hyperref[alg:neighbourhood:text_initialisation]{Line~\ref*{alg:neighbourhood:text_initialisation}}). Next, for every subsumption relation having the input concept $C$ in its signature, either $C's$ name or the anchor node's name is appended first. For example, given the following subsumption relations $\{Car \sqsubseteq Vehicle, Cabrio \sqsubseteq Car\}$, the algorithm generates $T=\{$\textit{"Every Car is a Vehicle"}$,$ \textit{"Every Cabrio is a Car"}$\}$ under the assumption that \textit{Car} was chosen as anchor node.


% SECTION: EMBEDDED CONTEXT %
\section{Embedded Context}\label{sec:embedded_context}
In this section we describe another approach of generating context descriptions based on semantic metadata. For that, we used \hyperref[sec:OWL_annotation_properties]{Annotation~Properties} which were defined as part of OWL. To maximise interoperability with existing libraries for ontology processing and manipulation we made use of the \hyperref[sec:dublin_core_metadata_vocabulary]{Dublin~Core~Metadata~Set}, a standard vocabulary designed to annotate resources with simple, textual information. A prerequisite for context generation is that presence such metadata information. However, as none of the ontologies we used for evaluation contained such metadata, we had to add them manually. 

This section starts by introducing annotation properties which are defined as part of the \hyperref[sec:OWL_annotation_properties]{Web~Ontology~Language~(OWL)\footnote{\url{https://www.w3.org/OWL/} accessed 2018/18/12}}. We used annotations to encode the context descriptions. Next, an overview of the \hyperref[sec:dublin_core_metadata_vocabulary]{Dublin~Core~Metadata~Set} is given as some parts were used for the definition of context properties. Then, in the remainder of this section our \hyperref[sec:enrichment_metaData_approach]{Metadata~based~Approach} is discussed.

\subsection{OWL Annotation Properties}\label{sec:OWL_annotation_properties}
Annotation properties first defined by OWL~1\footnote{\url{https://www.w3.org/TR/owl-ref/\#Annotations} accessed 2018/20/12} and then extended by OWL~2\footnote{\url{https://www.w3.org/TR/owl2-syntax/\#Annotation_Properties} accessed 2018/20/12} are used to enhance concepts, properties, individuals and ontology headers with meta data such as labels, comments, creation date and so forth. This information does not alter the semantics of the ontology in any way, it is merely intended for documentation purposes and therefore ignored by reasoning engines. 

Besides the built-in annotation properties OWL~1 also offers the ability to create user-defined annotation properties. An example of using \emph{owl:AnnotationProperty} to declare a user-defined annotation property is given in~\hyperref[lst:user_defined_annotation_property]{Listing~\ref*{lst:user_defined_annotation_property}}. In this example, the OWL~Class \emph{Lens} is annotated by the custom annotation property \emph{dc:date} which is defined by the \hyperref[sec:dublin_core_metadata_vocabulary]{Dublin~Core~Metadata~Set} discussed in the next section. 
\begin{lstlisting}[frame=single,caption=Declaration of user-defined annotation property in OWL~1,label=lst:user_defined_annotation_property]
	<rdf:RDF
		xmlns:rdf="http://www.w3.org/1999/02/22-rdf-syntax-ns#"
		xmlns:dc="http://www.purl.org/metadata/dublin-core#"
		xmlns:owl="http://www.w3.org/2002/07/owl#">
		
		<owl:AnnotationProperty 
			rdf:about="http://purl.org/metadata/dublin-core#date"/>
		
		<owl:Class rdf:about="http://www.photo.org/camera#Lens"
			<dc:date rdf:datatype="http://www.w3.org/2001/XMLSchema#date">
				2018-12-20
			</dc:date>
		</owl:Class>
		
	</rdf>
\end{lstlisting}

\subsection{Ontology Metadata Standards}\label{sec:ontology_metadata_standards}
Over the years ontologies were used in many domain contexts, including general-purpose as well as highly specialised ones. Obviously, what separates good ontologies from poor ones is how well they are documented~\cite{daquin2012}. Studies~\cite{dutta2017} analysed various approaches of embedding metadata in ontologies. The outcome was that there is no standard way to describe and document ontologies, albeit a few vocabularies that describe semantic metadata exist. Two of the most common vocabularies are briefly described next.

\paragraph{Dublin Core Metadata Set~(DC)}\label{sec:dublin_core_metadata_vocabulary} Being one of the most prominent vocabulary in describing semantic metadata, published and maintained by the Dublin Core Metadata Initiative~(DCMI), it originally contained 15 metadata terms\footnote{\url{http://www.dublincore.org/documents/dces/} accessed 2018/05/20},  designed to annotate resources with simple, textual information. Since its first launch, the project have gained popularity, including more than 127 terms\footnote{\url{http://www.dublincore.org/documents/dcmi-terms/} accessed 2018/05/20}. The initial set of terms is listed in~\hyperref[app:dc_terms]{Appendix~\ref*{app:dc_terms}}. 

To maximise interoperability in heterogeneous environments, an RDF-Schema with DCMI-Metadata\footnote{\url{http://dublincore.org/schemas/rdfs/} accessed 2018/05/20} elements was created, in which each entity is identified by a Uniform Resource Identifier~(URI) starting with the prefix \emph{http://purl.org}. A broader discussion on the use of metadata in general is given in~\cite{nilsson2010}.  

\paragraph{Simple Knowledge Organisation~(SKOS)}
The SKOS Core Vocabulary~\cite{skos2005} defines a set of RDF properties and RDFS classes
used to express the content and structure of a concept scheme, which describes sets of concepts with optionally linked concepts. The vocabulary is standardised by the W3C~Consortium\footnote{\url{https://www.w3.org/TR/skos-reference/} accessed 2018/05/20}. Relevant terms are listed in~\hyperref[app:skos_terms]{Appendix~\ref*{app:skos_terms}}.
	
There is some overlap between DC and SKOS. For example, the terms \textit{dc:subject} and \textit{skos:subject} describes similar characteristics of an entity. However, in some usage scenarios the range of skos:subject is limited to resources of type skos:concept compared to the unrestricted range of dc:subject. Moreover, in case there is more than one subject, the property \textit{skos:primarySubject} allows assertions of the entities or resources main subject. 


\subsection{Metadata based Approach}\label{sec:enrichment_metaData_approach}
Given the high number on ontology metadata formats from above, \hyperref[alg:embedded_enrichment]{Algorithm~\ref*{alg:embedded_enrichment}} shows the pseudocode to create concept descriptions extracted from embedded metadata. In addition to the notation used in the previous section we define $\Phi(C) \coloneqq \{m_1, m_2, \ldots, m_i \}$ where $m_i$ is the $i'th$ metadata element embedded in concept $C$ and $T$ is the description of some metadata element.

\begin{algorithm}
	\caption{Context Enrichment based on embedded metadata}\label{alg:embedded_enrichment}
	\begin{algorithmic}[1]
		\Procedure{Generate Description}{}\newline
			\textbf{Input:} A concept $C$ with embedded metadata $\{m_1, m_2, \ldots, m_i \}$\newline
			\textbf{Output:} A description $T$ of $C's$ metadata elements\newline
			\State{$T=\{\}$}
			\For {$ m_k \in \Phi(C) $}
				\State $T=T$ $\cup$ $m_k$
			\EndFor
		\EndProcedure
	\end{algorithmic}
\end{algorithm}

While the actual enrichment is straightforward, it collects all descriptions for a determined concept, the details of extracting the metadata from annotation properties is omitted here because it highly depends on the chosen metadata encoding.
As we decided to encode the metadata in annotation properties, the extraction process works by selecting the related annotation properties for a specified concept. 

To illustrate the concept of the context generation algorithm a simple example of an OWL~Class enriched with metadata is shown in~\hyperref[lst:metadata_based_rdf_example]{Listing~\ref*{lst:metadata_based_rdf_example}}.
For that, the algorithms generates $T=\{$ \emph{"Greenhouse gas (GHG) is one of several gases, especially carbon dioxide, that prevent heat from the earth escaping into space, causing the greenhouse effect. Greenhouse gases from human activities are the most significant driver of observed climate change since the mid-20th century.", "greenhouse gas"} $\}$. 
\hyperref[fig:metadata_rdf_example_questionaire]{Figure~\ref*{fig:metadata_rdf_example_questionaire}} depicts the questionnaire presented to crowd workers for the example from above. 

\begin{lstlisting}[frame=single,breaklines=true,postbreak=\mbox{\textcolor{black}{$\hookrightarrow$}\space},caption=An OWL Class enriched with metadata,label=lst:metadata_based_rdf_example]
	<rdf:RDF
		xmlns:rdf="http://www.w3.org/1999/02/22-rdf-syntax-ns#"
		xmlns:rdfs="http://www.w3.org/2000/01/rdf-schema#"
		xmlns:dc="http://www.purl.org/metadata/dublin-core#"
		xmlns:owl="http://www.w3.org/2002/07/owl#">
		
		<owl:Class rdf:about="http://www.climatechange.org/greenhouse_gas"
			<dc:description>
				Greenhouse gas (GHG) is one of several gases, especially carbon dioxide, that prevent heat from the earth escaping into space, causing the greenhouse effect. Greenhouse gases from human activities are the most significant driver of observed climate change since the mid-20th century.
			</dc:description>
			<rdfs:label>
				greenhouse gas
			</rdfs:label>
		</owl:Class>
		
	</rdf>
\end{lstlisting}

\begin{figure}
	 \centering
	 \includegraphics[width=\textwidth]{screenshots/questionaire_metadata_example}
	 \caption{Questionnaire presented to crowd workers for the OWL Class example}\label{fig:metadata_rdf_example_questionaire}
\end{figure}


% SECTION: EXTERNAL SOURCE %
\section{External Source}\label{sec:external_source}
An alternative method of context enrichment is based on fetching concept definitions from external sources, especially when these are not already available as metadata annotations in the ontologies that are validated. The lookup is solely based on the concept's name, neglecting the connected nature of an ontology. Dictionaries have always been the first choice when it comes to searching for specific information about words or phrases. We chose \textit{WordNik}\footnote{\url{https://www.wordnik.com/} accessed 2018/06/15} as source for external content, a freely available online dictionary for the English language. Among other features that were offered, we used \emph{example sentences} that were collected from various sources across the Web. 

This section begins with a brief introduction to WordNik, the online dictionary we used for the provision of example sentences, and then continues with our approach of using WordNik as content provider for concept descriptions.   


\subsection{WordNik}\label{sec:wordnik}
WordNik targets native English speakers who look up words that are rare~(technical terms or dialect terms), very old or very new. They often search for definitional information which is incomplete or missing in traditional dictionaries. Users tolerate published imperfection because they opt for relevant, actual and cutting-edge information, even though not officially approved by editors~\cite{burnett1979}. They want to understand the context of word usage in sentences, not necessarily explanatory statements as in printed or even online dictionaries.

The driving force behind WordNik was contribution. It processes and aggregates external user-generated content such as tweets, newspaper articles, scientific articles or uploaded Flickr\footnote{\url{https://www.flickr.com/} accessed 2018/06/15} images. This is similar to what search engines do, but with restricted scope. The creators of WordNik observed that very few people write word definitions, they rather add meta linguistic information such as lists of their favourite words, comments or tags. WordNik additionally collects statistics about lexicographical terms, more or less frequently searched words and most commented words. 

WordNik also offers an API for programmatically accessing their resources\footnote{\url{https://developer.wordnik.com/} accessed 2018/06/15}. Besides the word definition, it also provides access to audio metadata, etymology, word usage, syllable information, bi-gram phrases, text pronunciations, relation diagrams to other words, example sentences and others. At the time of writing this thesis free access is granted for non-profit, non-commercial use with a limitation on the number of API calls. 
After a successful registration process, an API token is provided which is a prerequisite for API interaction. Besides Web access, a handful of libraries\footnote{\url{https://developer.wordnik.com/libraries} accessed 2018/06/15}, available in many programming languages, were created to facilitate integration with third-party applications. 

\subsection{Dictionary based Approach}\label{sec:enrichment_dictionary_approach}
Intuitively, the idea of generating descriptions using dictionary lookups is simple: starting from a concept name, descriptions are built from consulting the online dictionary \hyperref[sec:wordnik]{WordNik}. 

A schematic overview of the overall workflow is shown in~\hyperref[fig:external_source_workflow]{Figure~\ref*{fig:external_source_workflow}} and described below:
\begin{figure}
	 \centering
	 \includegraphics[width=0.9\textwidth]{drawio/External_Source_Workflow}
	 \caption{Conceptual workflow of WordNik consultation to generate concept descriptions}\label{fig:external_source_workflow}
\end{figure}

\begin{enumerate}[label=\textbf{[Step \Roman*]},leftmargin=\widthof{[Step III]}+\labelsep]
	\item \emph{(Concept Selection)} The first step in the workflow is the selection of the concept(s)
	      for generating the description. For that, the ontology engineer selects the concepts and
		  starts the enrichment process. 
	\item \emph{(Text Normalisation)} The idea is to use the concept name as a baseline for any further
	      processing. Often, the name can not be used directly as input to WordNik because it
		  contains unwanted characters such as excessive spaces, quotes, dots or just non-printable
		  characters. This is especially true for learned ontologies, generated from textual sources.
		  Our algorithm uses the built-in text manipulation capabilities of the JDK to pre-process
		  concept names.
	\item \emph{(Dictionary Lookup)} Next, WordNik is consulted to find example sentences for normalised
	      concept names. In contrast to traditional dictionaries, WordNik searches in all kinds of available
		  online content, including newspapers, journals, scientific publications, tweets and others. All
		  API interaction is protected against unauthorised access, however, to help developers learning
		  the API, some features are available in isolated Sandbox~Mode\footnote{\url{https://developer.wordnik.com/docs} accessed  2018/06/21}.
		  
		  For example, when searching for the word \guillemotright chartjunk\guillemotleft~which does not
		  have a definition in traditional dictionaries, the API response is illustrated
		  in~\hyperref[lst:WordNik_response_for_chartjunk]{Listing~\ref*{lst:WordNik_response_for_chartjunk}}. The
		  output is encoded in JavaScript Object Notation~(JSON)\footnote{\url{https://tools.ietf.org/html/rfc7159} accessed 2019/01/05}
		  which defines a common, human-readable format for data transmission. The example shows one \emph{example
		  sentence}~(the others were omitted because they share the same structure) including various other
		  properties besides the actual title and text. Our algorithm just skips these other properties because they were
		  not needed for the final concept description. However, it might be useful in certain scenarios to differentiate 
		  duplicate entries by exampleId or exploring further details by adding the source URL.
	\item \emph{(Text Buffering)} Depending on wether a single concept or multiple concepts are validated, example sentences
	      need to be harmonised, which is realised by storing intermediate results and mapping these to the initial concepts.
	\item \emph{(Crowdsourcing Submission)} The last step of the workflow is the creation of the questionnaire for the actual
	      ontology validation. As for all enrichment methods, the only part that varies for each approach is the concept
		  description, shown as top part of the template.
		  \hyperref[fig:wordnik_example_questionaire]{Figure~\ref*{fig:wordnik_example_questionaire}} depicts the
		  questionnaire presented to crowd workers for the example from above.
\end{enumerate}

\begin{lstlisting}[frame=single,breaklines=true,postbreak=\mbox{\textcolor{black}{$\hookrightarrow$}\space},caption=WordNik API response for the word \guillemotright chartjunk\guillemotleft,label=lst:WordNik_response_for_chartjunk]
	{
	  "examples": [
	    {
	      "provider": {
	        "name": "spinner",
	        "id": 712
	      },
	      "year": 2008,
	      "rating": 185,
	      "url": "http://www.emersonprocessxperts.com/archives/2008/10/improving_how_y.html",
	      "word": "chartjunk",
	      "text": "Marshall described \"chartjunk\" as additional graphics not related to the data in a quest to make the chart more aesthetically pleasing.",
	      "title": "Emerson Process Experts",
	      "documentId": 15463705,
	      "exampleId": 289744774
	    },
		...
	  ]
	}
\end{lstlisting}

\begin{figure}
	 \centering
	 \includegraphics[width=\textwidth]{screenshots/questionaire_wordnik_example}
	 \caption{Questionnaire presented to crowd workers for searching \guillemotright chartjunk\guillemotleft~on WordNik}\label{fig:wordnik_example_questionaire}
\end{figure}
% To conclude, this approach is rather simple and easy to implement, however, it may have the potential to generate wrong results, especially for ambiguous concept names. 


% SECTION: SUMMARY %
\section{Summary}\label{sec:state_of_the_art_summary}
In this chapter we provided a brief overview of research fields that are relevant for our work.

In the beginning of this chapter we gave a brief introduction into Crowdsourcing~(\hyperref[sec:state_of_the_art_crowdsourcing]{Section~\ref*{sec:state_of_the_art_crowdsourcing}}). We briefly discussed the potentials and risks and listed some fields in which Crowdsourcing techniques have been successfully applied. 

Then, we focused on the interplay between the Semantic Web and Crowdsourcing~(\hyperref[sec:state_of_the_art_crowdsourcing_and_the_semantic_web]{Section~\ref*{sec:state_of_the_art_crowdsourcing_and_the_semantic_web}}). We presented the Semantic Web life cycle and gave examples of Crowdsourcing approaches for each stage of the life cycle. 

In~\hyperref[sec:ucomp_protege_plugin]{Section~\ref*{sec:ucomp_protege_plugin}} we presented the uComp Protege Plugin on which our implementation builds on. We described the plugin functionality and the supported ontology validation tasks. For the creation of Crowdsourcing tasks, we looked into the workflow of the plugin to facilitate ontology validation. 
Unfortunately, crowd workers often do not had enough knowledge to complete Crowdsourcing tasks. They need additional contextual information which improves their understanding. Before investigating our approaches which generate contextual information, we had to give a common definition of \guillemotright Context\guillemotleft:
Context refers to any sort of additional information that is supplied with a Crowdsourcing task to improve its understanding in
such a way that it positively affects the crowds performance and the result quality. Furthermore, we do not set a limitation on the type or format of Context that is provided. Even tough there exists some approaches that use Context in Crowdsourcing tasks,
they all use a different notion of Context.



%%%%%%%%%%%%%%%%%%%%%%%%%%%%%%%%%%%%%%%%%%%%%%%%%%%%%%%%%%%%%%%%%%%%%%%%%%%%%%%%%%%%%%%%%%%%%%%%%%%%%%%%%%%%%%%%%%%%%%%%%%%%%%%%%%%%%%%%%%%%%%%%%%%%