\section{External Source}\label{sec:external_source}
An alternative method of context enrichment is based on fetching concept definitions from external sources, especially when these are not already available as metadata annotations in the ontologies that are validated. The lookup is solely based on the concept's name, neglecting the connected nature of an ontology. Dictionaries have always been the first choice when it comes to searching for specific information about words or phrases. We chose \textit{WordNik}\footnote{\url{https://www.wordnik.com/} accessed 2018/06/15} as source for external content, a freely available online dictionary for the English language. Among other features that were offered, we used \emph{example sentences} that were collected from various sources across the Web. 

This section begins with a brief introduction to WordNik, the online dictionary we used for the provision of example sentences, and then continues with our approach of using WordNik as content provider for concept descriptions.   


\subsection{WordNik}\label{sec:wordnik}
WordNik targets native English speakers who look up words that are rare~(technical terms or dialect terms), very old or very new. They often search for definitional information which is incomplete or missing in traditional dictionaries. Users tolerate published imperfection because they opt for relevant, actual and cutting-edge information, even though not officially approved by editors~\cite{burnett1979}. They want to understand the context of word usage in sentences, not necessarily explanatory statements as in printed or even online dictionaries.

The driving force behind WordNik was contribution. It processes and aggregates external user-generated content such as tweets, newspaper articles, scientific articles or uploaded Flickr\footnote{\url{https://www.flickr.com/} accessed 2018/06/15} images. This is similar to what search engines do, but with restricted scope. The creators of WordNik observed that very few people write word definitions, they rather add meta linguistic information such as lists of their favourite words, comments or tags. WordNik additionally collects statistics about lexicographical terms, more or less frequently searched words and most commented words. 

WordNik also offers an API for programmatically accessing their resources\footnote{\url{https://developer.wordnik.com/} accessed 2018/06/15}. At the time of writing this thesis free access is granted for non-profit, non-commercial use with a limitation on the number of API calls. 
After a successful registration process, an API token is provided which is a prerequisite for API interaction. Besides Web access, a handful of libraries\footnote{\url{https://developer.wordnik.com/libraries} accessed 2018/06/15}, available in many programming languages, were created to facilitate integration with third-party applications. 


\subsection{Dictionary based Approach}\label{sec:enrichment_dictionary_approach}
Intuitively, the idea of generating descriptions using dictionary lookups is simple: starting from a concept name, descriptions are built from consulting an online dictionary. 

A schematic overview of the overall workflow is shown in~\hyperref[fig:external_source_workflow]{Figure~\ref*{fig:external_source_workflow}}.
\begin{figure}
	 \centering
	 \includegraphics[width=0.9\textwidth]{drawio/External_Source_Workflow}
	 \caption{Conceptual workflow of WordNik consultation to generate concept descriptions}\label{fig:external_source_workflow}
\end{figure}
The idea is to use the concept name as a baseline for any further processing. Often, the name can not be used directly as input to WordNik because it contains unwanted characters such as excessive spaces, quotes, dots or just non-printable characters. This is especially true for learned ontologies, generated from textual sources. Our algorithm uses the built-in text manipulation capabilities of the JDK to pre-process concept names. Among the other approaches introduced earlier in this section, it is yet simple but still provides good results. 

Next, WordNik is consulted to find example sentences for normalised concept names. In contrast to traditional dictionaries, WordNik searches in all kinds of available online content, including newspapers, journals, scientific publications, tweets and others. 
All API interaction is protected against unauthorised access, however, to help developers learning the API, some features are available in isolated Sandbox~Mode\footnote{\url{https://developer.wordnik.com/docs} accessed  2018/06/21}. For instance, example sentences for the word \texttt{bird} can be fetched from the URL \texttt{/word.json/bird/examples}.

Depending on wether a single concept or multiple concepts are validated, example sentences need to be harmonised, which is realised by storing intermediate results and mapping these to the initial concepts. 

To conclude, this approach is rather simple and easy to implement, however, it may have the potential to generate wrong results, especially for ambiguous concept names. 
