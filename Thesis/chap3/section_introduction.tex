\section{Introduction}\label{sec:approaches_introduction}
Previous experiments using the uComp~Protege~Plugin~\cite{wohlgenannt2016}, where this thesis builds up on, had successfully applied crowdsourcing techniques on ontology validation. They conclude that it leads to data quality comparable to that of manually performed validation done by ontology engineers while reducing the overall costs. However, there was still potential to improve the worker performance in terms of speed and quality. 
Studies~\cite{mortensen2013} confirmed this statement concluding that the best worker performance is achieved
\enquote{with questions formulated in the most basic form, a domain-specific qualification, and concept definitions for context}.

Based on the existing platform for creating crowdsourcing jobs within Protege, we fill this gap by investigating different approaches of context creation.
The \emph{first variant}~(\hyperref[sec:enrichment_ontology_approach]{\textbf{Ontology~based~Approach}}) uses the relations encoded by the ontology itself to generate textual definitions. At the current state, only subsumption~relations are considered, but the algorithm is rather generic which facilitates future adaptions~(e.g. including other relation types for context creation).
The \emph{second variant}~(\hyperref[sec:enrichment_metaData_approach]{\textbf{Meta-Data based Approach}}) depends on annotations embedded in the ontology which were manually added by ontology engineers. Among various metadata standards which define common meanings of annotation content, our approach is based on the~\hyperref[sec:dublin_core_metadata_vocabulary]{Dublin~Core} vocabulary.
For the \emph{third variant}~(\hyperref[sec:enrichment_dictionary_approach]{\textbf{Dictionary based Approach}}) the idea was to generate context by consulting an online dictionary. We decided in favour of~\hyperref[sec:wordnik]{WordNik}, which provides access to a large number of online content including tweets, newspaper articles and scientific articles. For the context creation, example sentences were fetched including the requested concept name. 