\section{Summary}\label{sec:state_of_the_art_summary}
In this chapter we provided a brief overview of research fields that are relevant for our work.

In the beginning of this chapter we gave a brief introduction into Crowdsourcing~(\hyperref[sec:state_of_the_art_crowdsourcing]{Section~\ref*{sec:state_of_the_art_crowdsourcing}}). We briefly discussed the potentials and risks and listed some fields in which Crowdsourcing techniques have been successfully applied. 

Then, we focused on the interplay between the Semantic Web and Crowdsourcing~(\hyperref[sec:state_of_the_art_crowdsourcing_and_the_semantic_web]{Section~\ref*{sec:state_of_the_art_crowdsourcing_and_the_semantic_web}}). We presented the Semantic Web Lifecycle and gave examples of Crowdsourcing approaches for each stage of the Lifecycle. 

In~\hyperref[sec:ucomp_protege_plugin]{Section~\ref*{sec:ucomp_protege_plugin}} we presented the uComp Protege Plugin on which our implementation builds on. We described the plugin functionality and the supported ontology validation tasks. For the creation of Crowdsourcing tasks, we looked into the workflow of the plugin to facilitate ontology validation. 
Unfortunately, crowd workers often do not had enough knowledge to complete Crowdsourcing tasks. They need additional contextual information which improves their understanding. Before investigating our approaches which generate contextual information, we had to give a common definition of \guillemotright Context\guillemotleft:
Context refers to any sort of additional information that is supplied with a Crowdsourcing task to improve its understanding in
such a way that it positively affects the crowds performance and the result quality. Furthermore, we do not set a limitation on the type or format of Context that is provided. Even tough there exists some approaches that use Context in Crowdsourcing tasks,
they all use a different notion of Context. Furthermore, none of these generate contextual information which is discussed in the main part of this thesis. 

