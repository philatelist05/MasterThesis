\section{The use of Context in Crowdsourcing Tasks}
% Notes from meeting with MS:
% 
% -) In related work, consider the use of context (e.g. define what "Context" means --> see definition from MS)
% -) Use table with references provided by MS

\subsection{Context}
When analysing the use of context enriched crowdsourcing tasks, we noticed that there exists no formal definition of that term. Instead, all approaches that were found use a different notion of context. 
This section first investigates what definitions of context these approaches were used. After that, we give a consolidated definition of context that fits our approach of crowd-based ontology validation.

When investigating the use of context in crowdsourcing tasks, a good start is to look at~\cite{sarasua2015crowdsourcing}. In this work, the authors did an extensive literature study to find challenges in the context of crowdsourcing and the Semantic Web. One of the challenges they found was a proper definition of context as part of a complete task design. Concretely, they asked but did not answer the minimum required context a crowd needs to finish a task correctly. Unfortunately, during our studies we could not find an answer either. It seems that there exists no generic answer which is applies in all contexts, it rather depends on the concrete type of task that needs to be solved. 


\subsection{Approaches that use Context for Ontology Validation}
% See table from MS of existing apporoaches that use Context in Human Computation %
