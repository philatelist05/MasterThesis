\section{Simple Knowledge Organisation~(SKOS)}
The SKOS Core Vocabulary~\cite{skos2005} defines a set of RDF properties and RDFS classes
used to express the content and structure of a concept scheme, which describes sets of concepts with optionally linked concepts. The vocabulary is standardised by the W3C~Consortium\footnote{\url{https://www.w3.org/TR/skos-reference/} accessed 2018/05/20}. Relevant terms are listed in~\hyperref[app:skos_terms]{Appendix~\ref*{app:skos_terms}}.
	
There is some overlap between DC and SKOS. For example, the terms \textit{dc:subject} and \textit{skos:subject} describes similar characteristics of an entity. However, in some usage scenarios the range of skos:subject is limited to resources of type skos:concept compared to the unrestricted range of dc:subject. Moreover, in case there is more than one subject, the property \textit{skos:primarySubject} allows assertions of the entities or resources main subject. 
