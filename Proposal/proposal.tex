\documentclass[12pt, notitlepage]{article}
\usepackage[margin=4cm]{geometry}
\usepackage{hyperref}
\usepackage[english]{babel}
\usepackage{tocloft}
\usepackage{setspace}
\usepackage{graphicx}
\usepackage{caption}
\usepackage{subcaption}
\usepackage{listings}
\usepackage{float}
\usepackage{tabularx}
\usepackage[ruled,vlined]{algorithm2e}
\usepackage[numbers]{natbib}
\usepackage[style=american]{csquotes}
\usepackage{paralist}

\setcounter{tocdepth}{4}
\cftsetindents{paragraph}{1cm}{0cm}


\title{TitlePage}
\author{Stefan Gamerith\\\\
		\emph{Student ID: 0925081}}

\begin{document}
	\maketitle
	\thispagestyle{empty}
	\newpage
\setcounter{page}{1}

\section{Problem Definition}
Knowledge bases handling vast amounts of Linked Data~(e.g. DBpedia~\cite{lehmann2015dbpedia}) gained importance, requiring links to large, but domain-specific datasets. Specifically managing the process of ontology engineering, including ontology creation and ontology evolution, tends to be complex and time-consuming if done by ontology experts.
\citeauthor{wohlgenannt2016crowd}~\cite{wohlgenannt2016crowd}~proposed the uComp Protege plugin for decomposing these tasks into simple and smaller ones, solvable by non-experts. Crowdsourcing methods were used to verify \textit{Relation Correctness~(T2)}, \textit{Relation Type~(T3)} and \textit{Domain Relevance~(T4)}, whereas Term Relatedness~(T1) was not supported. Although evaluation shows that the tool greatly reduces working time by ontology engineers, there were some challenges and limitations. 

This work investigates one such limitation as the lack of contextual information in the ontology engineering tasks. More specifically, it investigates the questions of what kind of additional context is needed such that the results get more accurate and to what extend does it improve the overall workflow. In addition, options for adding task specific contexts, a detailed architecture for an implementation and and evaluation against a set of pre-selected ontologies are subjects of contribution. 
\section{Expected Result}
Based on the problem definition above, the following research questions are addressed in this thesis:
\paragraph{RQ1}~\textbf{What are the requirements for an extension of the uComp Protege plugin adding contextual information to crowdsourcing tasks?}\\
\paragraph{RQ2}~\textbf{What approaches of different research areas are applicable to facilitate the provision of contextual information for crowdsourcing tasks?}\\
\paragraph{RQ3}~\textbf{Does the proposed extension outperform the existing one on a selected set of ontologies?}\\
This thesis make use of empirical and exploratory approaches strengthening the confidence in answering the research questions. The following findings as the result of applying the proposed methodology are planned:
\begin{itemize}
	\item Functional and non-functional requirements for building an extension of the uComp Protege plugin.
	\item A detailed study of various options for enhancing crowdsourcing tasks with contextual information.
	\item A designed, implemented and evaluated extension of the uComp Protege plugin.
	\item An in-depth evaluation on the performance of the proposed extension.
	\item A reduction in costs and working time in crowdsourcing tasks while improving quality results from crowds.
\end{itemize}
\section{Methodology and Approach}
The expected results are derived from the following methodology and approach:
\subsection{Literature Research}
Some text here.
\subsection{Identify requirements}
Some text here.
\subsection{Identify different approaches for the implementation}
Some text here.
\subsection{Integrate approach in the uComp Protege plugin}
Some text here.
\subsection{Evaluate performance metrics}
Some text here.

\section{State of the art}
\section{Relation to Software Engineering \& Internet Computing}



\newpage
\bibliography{literature}
\bibliographystyle{plainnat}

\end{document}