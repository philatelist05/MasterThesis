\section{The uComp Protege Plugin}\label{sec:ucomp_protege_plugin}
In this section we present the uComp Protege Plugin~\cite{wohlgenannt2016} where our implementation builds up on. The Plugin was realised as a plugin for the Protege\footnote{\url{https://protege.stanford.edu/} accessed 2018/08/07} ontology editor. It enables the automatic creation of Crowdsourcing tasks to support ontology validation. It was designed as a tool used by ontology engineers to reduce the burden of manual ontology validation and maintenance.

\subsection{Plugin Functionality}
The plugin supports the following tasks which were previously done by experts in collaboration with domain experts:

\paragraph{Verification of Domain Relevance}
The goal of this task is to decide whether a given concept~(or a set of concepts) is relevant for a given domain. For that, crowd workers need to
answer a binary question, that is a question with a yes/no answer. The corresponding crowdsourcing task is automatically generated by the platform and
contains besides the actual concept that should be validated also the domain and optionally some additional information that is useful for answering 
the question correctly. 

\paragraph{Verification of Relation Correctness}
Judging the correctness of relations, the plugin offers interfaces that allow the validation of subsumption and instanceOf relations. 
For subsumption, the crowd needs to decide whether a given concept is a subclass of another concept. For example, validating the correctness
of the subclass relation \emph{isSubClass(weather, rain)} for the domain \emph{climate change}, crowd workers have to decide whether the concept rain is a sub-class~(sub-concept) of the concept weather in the domain of climate change. Validating the correctness of instanceOf relations, the crowd needs to decide whether a given individual is an instance of a given concept~(class). For example, contributors where asked to decide whether the individual \emph{Bordeaux Region} is an instance of the concept \emph{Region} for the \emph{Wine} domain. 

\paragraph{Specification of Relation Type}
This task is different from the others described above. Instead of answering binary questions, the crowd was asked to assign relation types to unlabeled object properties. A prerequisite for this task is that object properties that are selected for evaluation were previously labelled as \emph{relation}. This way, the plugin knows which object properties take part in the validation process. Additionally, crowd workers can optionally suggest a new relation type if none of the suggested ones fit their needs. 

\paragraph{Verification of Domain and Range}
The purpose of this task is mainly to identify problems that are relevant for reasoning rather than validating the ontology structure itself. In this task, the crowd were asked to validate domain and range restrictions as specified by OWL. For example, the crowd needs to decide whether the object property \emph{hasSister} maps a person~(domain) to a female~(range). As stated earlier, errors in range and domain restrictions have no impact on the ontology itself but are rather used by reasoners to infer additional knowledge. 
