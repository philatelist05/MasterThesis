\documentclass[12pt, notitlepage]{article}
\usepackage[margin=4cm]{geometry}
\usepackage{hyperref}
\usepackage[english]{babel}
\usepackage{tocloft}
\usepackage{setspace}
\usepackage{graphicx}
\usepackage{caption}
\usepackage{subcaption}
\usepackage{listings}
\usepackage{float}
\usepackage{tabularx}
\usepackage[ruled,vlined]{algorithm2e}
\usepackage[numbers]{natbib}
\usepackage[style=american]{csquotes}
\usepackage{paralist}

\setcounter{tocdepth}{4}
\cftsetindents{paragraph}{1cm}{0cm}


\title{TitlePage}
\author{Stefan Gamerith\\\\
		\emph{Student ID: 0925081}}

\begin{document}
	\maketitle
	\thispagestyle{empty}
	\newpage
\setcounter{page}{1}

\section{Problem Definition}
Knowledge bases handling vast amounts of Linked Data~(e.g. DBpedia~\cite{lehmann2015dbpedia}) gained importance, requiring links to large, but domain-specific datasets. Specifically managing the process of ontology engineering, including ontology creation and ontology evolution, tends to be complex and time-consuming if done by ontology experts.
\citeauthor{wohlgenannt2016crowd}~\cite{wohlgenannt2016crowd}~proposed the uComp Protege plugin for decomposing these tasks into simple and smaller ones, solvable by non-experts. Crowdsourcing methods were used to verify \textit{Relation Correctness~(T2)}, \textit{Relation Type~(T3)} and \textit{Domain Relevance~(T4)}, whereas Term Relatedness~(T1) was not supported. Although evaluation shows that the tool greatly reduces working time by ontology engineers, there were some challenges and limitations. 

This work provides an extension to the Protege plugin by adding additional information to the tasks. In particular, it provides options for adding task specific contexts, presents the architecture for an implementation of one such option and evaluates the contribution against a set of pre-selected ontologies.
\section{Expected Result}
\section{Methodology and Approach}
\section{State of the art}



\newpage
\bibliography{literature}
\bibliographystyle{plainnat}

\end{document}