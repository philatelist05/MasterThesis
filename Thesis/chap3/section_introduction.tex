\section{Introduction}\label{sec:approaches_introduction}
Previous experiments using the uComp~Protege~Plugin~\cite{wohlgenannt2016} where this thesis builds up on, had successfully applied crowdsourcing techniques on ontology validation. They conclude that it leads to data quality comparable to that of manually performed validation done by ontology engineers while reducing the overall costs. However, there was still potential to improve the worker performance in terms of speed and quality. 
Studies~\cite{mortensen2013} confirmed this statement concluding that the best worker performance is achieved
\enquote{with questions formulated in the most basic form, a domain-specific qualification, and concept definitions for context}.

Based on the existing platform for creating crowdsourcing jobs within Protege, we fill this gap by investigating different approaches of context creation. 